\subsection{补充知识}
\begin{frame}[standout] 补充知识 \end{frame}
\begin{frame}{补充知识}
    \begin{myoutline}
        \1 下划线的 6 个作用
            \2 用在 Python 解释器,表示上一次的执行结果
            \2 代码中一个独立的下划线,表示这个变量不重要
            \2 类的内部,双下划线作为变量名或函数名的开头,表示私有
            \2 双下划线开头和结尾的方法,是魔术方法
                \3 比如常见的 `\_\_init\_\_', `\_\_dict\_\_', `\_\_dir\_\_', `\_\_doc\_\_', `\_\_eq\_\_' 等等
            \2 作为变量名中间的一部分
            \2 作为数字中间的一部分,更易读
    \end{myoutline}
\end{frame}
\begin{frame}{补充知识}
    \begin{myoutline}
        \1 Jupyterlab扔后台?conda/brew/apt install tmux
        \1 使用服务器的Jupyterlab
        \1 服务器(没有root权限)上安装的jupyterlab 后 自动加载的网页里不能显示notebook是啥(IP原因)
    \end{myoutline}
\end{frame}
\begin{frame}[fragile]{补充知识}
    \begin{myoutline}
        \1 Windows 配置 scoop 及 scoop 安装软件和命令
            \2 直接演示
    \end{myoutline}
    \begin{lstlisting}
# 进入powershell
set-alias ll  # ls
# 步骤 1:在 PowerShell 中打开远程权限
Set-ExecutionPolicy RemoteSigned -scope CurrentUser;
cd ~
# 步骤 2: 自定义 Scoop 安装目录
# 如果跳过该步骤, Scoop 将默认把所有用户安装的 App 和 Scoop 
# 本身置于C:\Users\user_name\scoop
# C:\Users\vagrant
mkdir scoop
$env:SCOOP='C:\Users\vagrant\scoop'
[Environment]::SetEnvironmentVariable('SCOOP', $env:SCOOP, 'User')

# 步骤 3:下载并安装 Scoop
iwr -useb https://gitee.com/glsnames/scoop-installer/raw/master/bin/install.ps1 | iex
scoop config SCOOP_REPO 'https://gitee.com/glsnames/scoop-installer'  # 设置镜像(如果软件安装失败的话)
scoop install git  # 必须的依赖项
    \end{lstlisting}

\end{frame}
\begin{frame}[fragile]{补充知识}
    \begin{lstlisting}
# step1: 添加官方维护的extras库(含大量GUI程序)
# 国内源 scoop bucket add extras https://gitee.com/scoop-bucket/extras.git
scoop bucket add extras
# step2 更新源
scoop update
# step3 安装 App和命令行工具(必装工具, conda装好可以不再装了)
scoop install git
scoop install miniconda3  # 安装全完全卸载原来安装的 miniconda!
scoop install sudo  # 调用管理员权限
scooop install tldr  # too long dont read!
# (类 unix 完美,windows 能用但信息提供的是类 unix 的,对 windows 不一定能够完全适用)
# 类unix 可以conda install tldr
scoop install busybox  # 项目实战一会用到! zcat命令的依赖
scoop install cwrsync  # 项目实战一会用到!
scoop install windows-terminal
scoop install powertoys


# 管理:
scoop list  # 查看已安装程序
scoop status  # 查看更新
scoop cleanup  # 删除旧版本
scoop checkup  # 自身诊断
    \end{lstlisting}
\end{frame}
