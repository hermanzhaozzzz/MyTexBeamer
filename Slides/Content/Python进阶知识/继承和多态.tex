\subsection{继承和多态}
\begin{frame}[standout] 继承和多态 \end{frame}
\begin{frame}[fragile]{把大象装进冰箱!--OOP,伪代码}
    \begin{columns}
        \column{0.5\textwidth}
        \begin{lstlisting}
class Box():
    """盒子类,实现了开门、关门方法"""

    def open_door(self):
        pass

    def close_door(self):
        pass

class IceBox(Box):
    """冰箱"""

    def ice(self):
        """制冷"""
        pass

class WaterBox(Box):
    """洗衣机"""
    
    def add_water(self):
        """加水"""
            pass
        \end{lstlisting}
        \column{0.5\textwidth}
        \begin{lstlisting}
    def sub_water(self):
        """排水"""
        pass   

    def wash(self):
        """洗涤"""
        pass

a = "大象"
ice_box = IceBox()   # 冰箱对象
ice_box.open_door()  # 通知冰箱开门
push(a)   # 推大象进入
ice_box.close_door()  # 通知冰箱关门


# 那我想关老虎呢?
b = "老虎"
ice_box.open_door()  # 通知冰箱开门
push(b)   # 推老虎进入
ice_box.close_door()  # 通知冰箱关门
            \end{lstlisting}
    \end{columns}
\end{frame}
% \2 方法重写
% \2 继承: 即一个派生类(derived class)继承基类(base class)的字段和方法。
% 继承也允许把一个派生类的对象作为一个基类对象对待。例如, 有这样一个设计: 一个Dog类型的对象派生自Animal类

% \1 继承: 即一个派生类(derived class)继承基类(base class)的字段和方法。
% \2 继承也允许把一个派生类的对象作为一个基类对象对待。
%     \3 例如, 有这样一个设计: 一个Animal类型的对象派生自Life类

% 多态