\subsection{代码规范}
\begin{frame}[standout] 代码规范 \end{frame}
\begin{frame}[fragile]{PEP8规范(常用的列举)}
    \begin{myoutline}
        \1 代码布局
            \2 每个缩进级别使用4个空格;连续行使用垂直对齐或者使用悬挂式缩进(额外的4个空格缩进)
            \2 空格是首选的缩进方法
            \2 每行最多79个字符
            \2 二元运算符前后换行都允许,只要代码保持一致就行。对于新代码建议在\textcolor{red}{二元运算符前进行换行}
            \2 空白行:使用\textcolor{red}{两个空白行}分隔\textcolor{red}{顶层函数和类定义};\textcolor{red}{类方法定义使用一个空行}分隔;使用额外的空白行来分隔相关逻辑功能
            \2 文件应该使用UTF-8编码, 且不应该有\textcolor{red}{编码声明}
            \2 导入多个库函数应该分开依次导入;导入总是放在文件的顶部,在任何\textcolor{red}{模块注释和文档字符串之后},在模块全局变量和常量之前;导入应按以下顺序进行:标准库导入、有关的第三方库进口、本地应用程序/库特定的导入,每组导入直接用空行分隔;避免通配符导入(import \*)
        \1 字符串
            \2 单引号字符串和双引号字符串相同,代码保持一致即可
            \2 对于三引号字符串,\textcolor{red}{常用三个双引号作文档字符串},文档字符串常用在模块的开端用以说明模块的基本功能,或紧跟函数定义的后面用以说明函数的基本功能
        \1 空格
            \2 避免使用无关的空格,包括空格内、逗号分号前面等; 避免在行末使用空格
            \2 二元运算符在两侧使用一个空格
            \2 当用于指示\textcolor{red}{关键字参数或默认参数值}时,不要在=符号周围使用空格
    \end{myoutline}
    \footnotenoindex{https://www.cnblogs.com/tangjielin/p/16511066.html}
    \footnotenoindex{https://peps.python.org/pep-0008/}
\end{frame}

\begin{frame}[fragile]{PEP8规范(常用的列举)}
    \begin{myoutline}
        \1 使用尾部逗号(trailing commas)
            \2 尾部逗号通常可选,除了用来说明是只有一个元素的元组tuple时
            \2 当参数、值等列表期望经常扩展时,通常是每个值一行,再加上一个尾部逗号
        \1 注释
            \2 代码更改时,相应的注释也要随之高优更改
            \2 注释应该是\textcolor{red}{完整的语句},第一个单词应该大写,除非它是特定标识符
            \2 \textcolor{red}{块注释}:缩进到与该代码相同的级别。块注释的每一行都以#和一个空格开始
            \2 \textcolor{red}{行注释}:对某一语句行进行注释,注释应该与语句至少隔开\textcolor{red}{两个空格},用#和一个空格开始
            \2 对于公共的modules, functions, classes, and methods,需要写文档字符串
            \2 注释应该是完整的语句,第一个单词应该大写,除非它是特定标识符
        \1 命名约定
            \2 python命名规范有点混乱,很难完全保存一致。对于新模块和包,应该遵守这些新的约定,已存在的库内部一致性更重要
            \2 命名应该\textcolor{red}{反应其用途而非实现}
            \2 不要将字符l(小写字母l),O(大写字母o)或I(大写字母I)作为单个字符变量名称
            \2 \textcolor{red}{模块名应该使用简短、全小写}的名字
            \2 \textcolor{red}{类的命名采用大驼峰命名法},即每个单词的首字母大写
            \2 \textcolor{red}{函数名称应该是小写的,为了提高可读性,必须使用由下划线分隔的单词}
    \end{myoutline}
    \footnotenoindex{https://www.cnblogs.com/tangjielin/p/16511066.html}
    \footnotenoindex{https://peps.python.org/pep-0008/}
    \footnotenoindex{https://google.github.io/styleguide/pyguide.html}
\end{frame}
\begin{frame}[fragile]{Google Python命名规范(常用的列举)}
    \begin{myoutline}
        \1 命名
            \2 异常名: ExceptionName ;
            \2 函数名: function\_name ;
            \2 全局常量名: GLOBAL\_CONSTANT\_NAME ;
            \2 全局变量名: global\_var\_name ;
            \2 实例名: instance\_var\_name ;
            \2 函数参数名: function\_parameter\_name ;
            \2 局部变量名: local\_var\_name .
            \2 函数名,变量名和文件名应该是描述性的,尽量避免缩写,特别要避免使用非项目人员不清楚难以理解的缩写,不要通过删除单词中的字母来进行缩写. 始终使用 .py 作为文件后缀名,不要用破折号.
        \1 命名约定
            \2 所谓”内部(Internal)”表示仅模块内可用, 或者, 在类内是保护或私有的。
            \2 用单下划线(\_)开头表示模块变量或函数是protected的(使用from module import \*时不会包含)。
            \2 用双下划线(\_\_)开头的实例变量或方法表示类内私有。
            \2 将相关的类和顶级函数放在同一个模块里. 不像Java, 没必要限制一个类一个模块.对类名使用大写字母开头的单词(如CapWords, 即Pascal风格), 但是模块名应该用小写加下划线的方式(如lower\_with\_under.py). 尽管已经有很多现存的模块使用类似于CapWords.py这样的命名, 但现在已经不鼓励这样做, 因为如果模块名碰巧和类名一致, 这会让人困扰。
    \end{myoutline}
    \footnotenoindex{https://google.github.io/styleguide/pyguide.html}
\end{frame}

\begin{frame}[fragile]{Google Python命名规范(常用的列举)}
    \begin{columns}
        \column{0.6\textwidth}
        \begin{myoutline}
            \1 `\_\_main\_\_'和main()
                \2 即使是一个打算被用作脚本的文件, 也应该是可导入的. 并且简单的导入不应该导致这个脚本的主功能(main functionality)被执行, 这是一种副作用. 主功能应该放在一个main()函数中.
                \2 在Python中, pydoc以及单元测试要求模块必须是可导入的. 你的代码应该在执行主程序前总是检查 if \_\_name\_\_ == `\_\_main\_\_' , 这样当模块被导入时主程序就不会被执行.
                \2 若使用 absl, 请使用 app.run\dots 所有的顶级代码在模块导入时都会被执行. 要小心不要去调用函数, 创建对象, 或者执行那些不应该在使用pydoc时执行的操作.
        \end{myoutline}
        \column{0.3\textwidth}
        \begin{lstlisting}
from absl import app 
... 
def main(argv): 
    # process non-flag arguments
    ... 


if __name__ == '__main__': 
    app.run(main)

# 否则,使用:
def main():
    ... 


if __name__ == '__main__': 
    main()
        \end{lstlisting}
    \end{columns}
    \footnotenoindex{https://google.github.io/styleguide/pyguide.html}
\end{frame}

\begin{frame}[fragile]{Google Python命名规范(常用的列举)}
    \begin{myoutline}
        \1 类型注释
            \2 通用规则请先熟悉下PEP-484对于方法
            \2 仅在必要时才对 self 或 cls 注释(实战课三当中进行讲解)
            \2 若对类型没有任何显示, 请使用 Any
            \2 无需注释模块中的所有函数
            \2 公共的API需要注释
            \2 在代码的安全性,清晰性和灵活性上进行权衡是否注释
            \2 对于容易出现类型相关的错误的代码进行注释
            \2 难以理解的代码请进行注释
            \2 若代码中的类型已经稳定,可以进行注释
            \2 对于一份成熟的代码,多数情况下,即使注释了所有的函数,也不会丧失太多的灵活性.
    \end{myoutline}
    \footnotenoindex{https://google.github.io/styleguide/pyguide.html}
\end{frame}