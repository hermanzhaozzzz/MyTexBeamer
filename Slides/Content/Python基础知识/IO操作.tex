\subsection{IO操作}
\begin{frame}[standout] IO操作 \end{frame}
\begin{frame}{IO操作-文件打开或创建}
    在 Python 中使用文件的关键函数是 open() 函数。
    open() 函数有两个参数:文件名和模式。
    \begin{myoutline}
        \1 文件名
        \1 模式
            \2 文件打开模式
                \3 r 读取 - \textcolor{red}{默认值}。打开文件进行读取,如果文件不存在则报错。
                \3 a 追加 - 打开供追加的文件,如果不存在则创建该文件。
                \3 w 写入 - 打开文件进行写入,如果文件不存在则创建该文件。
                \3 x 创建 - 创建指定的文件,如果文件存在则返回错误。
            \2 读作文本或二进制
                \3 t 文本 - \textcolor{red}{默认值}。文本模式。
                \3 b 二进制 - 二进制模式(例如图像)。
    \end{myoutline}
\end{frame}

\begin{frame}{IO操作-文件写入和读取}
    \begin{myoutline}
        \1 open() 函数返回文件对象
        \1 写入: 文件对象的 write 方法
        \1 读取: 文件对象的 read 方法用于读取文件的内容
            \2 read() 默认情况下, read() 方法返回整个文本,也可以指定要返回的字符数
            \2 readline() \textcolor{red}{readline() 方法每次}返回一行(iterator)
            \2 readlines() 把文件当做 list 返回,一行为一个元素
    \end{myoutline}


\end{frame}

\begin{frame}[fragile]{IO操作-文件关闭和删除}
    \begin{lstlisting}
import os

os.remove("demofile.txt")  \# 删除文件
os.rmdir("myfolder")  \# 删除文件夹
    \end{lstlisting}
\end{frame}