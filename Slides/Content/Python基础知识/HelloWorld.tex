\subsection{Hello World!}

\begin{frame}[standout] 第三节 \quad Python 基础知识 \end{frame}

\begin{frame}{Hello World}
    \begin{myoutline}
        \1 提示: 输入法使用英文!
        \1 两种运行方式
            \2 python 的shell页面
                \3 ``交互式'', 输入一行, 按回车, 返回一次结果, exit()退出(演示)
                \3 应用" 在 python 中测试少量代码
            \2 python 源代码文件``.py''
                \3 编辑好每一行代码, 在命令行全部运行(演示)
                \3 本质: 使用安装在\textcolor{red}{指定路径下的} python \textcolor{red}{解释器} 运行指定目录下的``.py''源代码文件(文本文件)(演示)
            \2 源代码文件本质是``文本文件''(可以使用 cat 或者 type 命令打印到命令行)
        \1 知识点
            \2 ``123'' 字符串
            \2 123 整数
            \2 \# 注释
                \3 注释一行
                \3 行内注释
        \1 使用不同 IDE 进行演示并说明不同 IDE 运行逻辑
    \end{myoutline}
\end{frame}

% Windows
% (Python Shell)
% >>> print("Hello Python!")
% Hello Python!
% >>> print("Hello World!")
% Hello World!
% >>> print("Hello Windows!")
% Hello Windows!
% MacOS
% >>> print("Hello MacOS!")
% Hello MacOS!
% Linux
% >>> print("Hello Linux!")
% Hello Linux!

% Windows
% python 源代码文件``.py''
% cat .py
% print("Hello World!")
% print("Hello Python!")
% print(1 + 1)

% python test.py
% Hello World!
% Hello Python!
% 2
% MacOS

% PyCharm
% 新建项目, 选择解释器
% /usr/local/Caskroom/miniconda/base/bin/python /Users/zhaohuanan/Downloads/test_PyCharm/test.py 
% Hello PyCharm!

% Jupyter Lab 缩放 150%
% print("Hello Jupyter Lab!")
% 打印解释器位置

% Jupyterlab 笔记从这里开始!!!


% # Hello World

% ## 查看解释器位置
% 需要使用 sys 模块
% `sys.executable`是一个变量,记录了正在使用的 Python 解释器位置


% ```python
% import sys

% sys.executable
% ```

% ## Hello World in JupyterLab

% ## print
% - print是一个函数名
% - print()调用这个函数
% - print 函数的功能是将括号中的内容打印到命令行(Jupyterlab 中是打印到 Cell 下面)


% ```python
% print("Hello Jupyter Lab!")
% ```


% ```python
% print("123")
% print(123)
% # print(123)
% ```

% ## 注释
% - 在 Python 中,使用 `#` 进行注释


% ```python
% # 1
% ```


% ```python
% 1
% ```

% ## 字符串和数字
% - 使用`'', "", """"""`括起来的内容,为字符串
% - 像 `123` 这种的,就是整数, `123.45` 为浮点数, 它们都是数值型的变量

% ## 补充知识
% - 使用`""""""`三引号进行多行注释
% - 使用`\`续航符对代码进行换行


% ```python
% # 多行注释,没有进行赋值操作,所以可以当做注释用
% """
% 123
% 231
% 231
% """
% ```

% ## 使用 Jupyterlab 进行基础知识的教学
% - 前提!
%       - Jupyterlab 默认保存为`.ipynb`的格式,类似`Rmarkdown.Rmd`格式, 而不是`.py`, 但是可以导出`.py`
%       - 经过了前面的演示大家知道了`.py` Python 源代码的运行机制
%       - Python 基础知识部分会使用 Jupyterlab 及 ipynb 的笔记本格式进行代码练习和笔记, 到 Python 进阶知识,我们会使用 PyCharm 这个 IDE 以及`.py`文件进行练习
% - 理由!
%       - Notebook 的形式利于整理知识点, 方便边学习知识点边练习
% - 提示!
%       - Jupyterlab 默认将单元格(Cell)的最后一行进行输出
%       - `;`可以使单元格(Cell)的最后一行不输出


% ```python
% 1 + 1
% ```


% ```python
% 1 + 1;
% ```

% 演示导出`.py`


% Python中的换行

\begin{frame}[standout]{开讲之前!}
    \begin{myoutline}
        \1 Python 基础
            \2 Jupyterlab学习环境
            \2 实战课, 基础部分知识点(全面覆盖) + 练习
            \2 在直播课程中缓冲的时间较少(每个练习和习题的用意)
            \2 在录播课程中反复观看和揣摩
        \1 Python 进阶
            \2 安装 PyCharm
            \2 适应 PyCharm 开发环境, 代码规范
            \2 逐步向本课程实战部分过度
    \end{myoutline}
\end{frame}
% 尽量全面覆盖