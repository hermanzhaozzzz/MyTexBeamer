\subsection{安装和配置JupyterLab环境}

\begin{frame}[standout]{安装和配置JupyterLab环境}
    \begin{myoutline}
        \1 使用Conda 安装和配置JupyterLab
            \2 
        \1 JupyterLab 基本用法
            \2 ipynb
            \2 terminal
    \end{myoutline}
\end{frame}

\begin{frame}[standout]install JupyterLab--Windows\end{frame}

\begin{frame}[fragile]{install JupyterLab}
    \begin{columns}
        \column{0.45\textwidth}
        \begin{lstlisting}
conda install jupyterlab nodejs # nodejs > 12.14
pip install jupyterlab_theme_hale # theme
jupyter-lab --no-browser --port 8888 # run
# 启用extension manager

# func: Settings Editor Form UI
# setting:
{
    "settingEditorType": "json"
}
# func: File Browser
# setting: 
{
    "navigateToCurrentDirectory": true,
    "useFuzzyFilter": true,
    "showLastModifiedColumn": true,
    "showHiddenFiles": true,
}
        \end{lstlisting}
        \column{0.45\textwidth}
        \begin{lstlisting}
# func: Text Editor
# setting:
{
    "editorConfig": {
        "autoClosingBrackets": true, // 补全括号
        "codeFolding": true, // 代码折叠
        "cursorBlinkRate": 530,
        "fontFamily": null,
        "fontSize": null,
        "insertSpaces": true,
        "lineHeight": null,
        "lineNumbers": true,
        "lineWrap": "on",
        "matchBrackets": true,
        "readOnly": false,
        "tabSize": 4,
        "rulers": [],
        "showTrailingSpace": false,
        "wordWrapColumn": 80
    }
}
        \end{lstlisting}
    \end{columns}
\end{frame}

\begin{frame}[standout]install JupyterLab--MacOS\end{frame}

\begin{frame}{已经整合的优秀插件}
    \begin{figure}
        \centering
        \includegraphics[width=0.99\linewidth]{Images/jupyterlabextension.jpg}
    \end{figure}
\end{frame}

\begin{frame}[fragile]{自动补全插件}
    \begin{columns}
        \column{0.45\textwidth}
        \begin{lstlisting}
pip install jupyterlab-lsp
pip install python-lsp-server pyright
conda install r-languageserver # 如果需要用 R


# func: Code Syntax 
# 启用或者关闭 lsp 自动补全
# true 为关闭,false 为开启
# setting:
{
    "disable": false,
}
        \end{lstlisting}
        \column{0.45\textwidth}
        \begin{lstlisting}
# func: Language Server
# setting:
{
    "language_servers": {
    "pyls": {
        "serverSettings": {
        "pyls.plugins.pydocstyle.enabled": true,
        "pyls.plugins.pyflakes.enabled": false,
        "pyls.plugins.flake8.enabled": true
        }
    },
    # 如果需要用 R
    "r-languageserver": {
        "serverSettings": {
        "r.lsp.debug": false,
        "r.lsp.diagnostics": false
        }
    }
    }
}
        \end{lstlisting}
    \end{columns}
\end{frame}
