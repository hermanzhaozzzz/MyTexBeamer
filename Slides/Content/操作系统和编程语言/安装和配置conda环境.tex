\subsection{安装和配置conda环境}
\begin{frame}{Python的包管理工具}
    \begin{myoutline}
        \1 pip
            \2 查询包
            \2 安装包
            \2 卸载包
            \2 一定程度的自动配置环境依赖功能
        \1 venv
            \2 创建 Python 的虚拟环境
            \2 其余功能类似 pip
        \1 conda
            \2 查询、安装、卸载Python包
            \2 创建、切换、管理Python运行环境
            \2 命令行工具安装(生信、数据科学必会工具)
            \2 强大的自动配置环境依赖功能
    \end{myoutline}
\end{frame}

\begin{frame}{Conda简介}
    \begin{columns}
        \column{0.3\textwidth}
        \begin{myoutline}
            \1 Conda是一款环境管理工具
                \2 最流行的 Python 环境管理工具之一
                \2 开源的\textcolor{red}{软件包管理系统}和\textcolor{blue}{环境管理系统},
                    \3 用于安装多个版本的软件包及其依赖关系, 并在不同环境间切换
                    \3 Conda 是为 Python 程序创建的, 也可以打包和分发其他软件
                \2 Linux, MacOS 和Windows跨平台
            \1 Conda
                \2 Anaconda
                \2 Miniconda
        \end{myoutline}
        \column{0.7\textwidth}
        \begin{figure}
            \begin{flushright}
                \includegraphics[width=0.5\linewidth]{Images/conda.jpg}

                \includegraphics[width=0.9\linewidth]{Images/conda-vs-miniconda-vs-anaconda.png}
            \end{flushright}
        \end{figure}
    \end{columns}
    \footnotenoindex{https://www.zhihu.com/question/369468216/answer/997004544}
\end{frame}
% \2 miniconda windows 64位安装包大小约为55 Mb, 只包含了conda、python、和一些必备的软件工具
% \2 anaconda windows 64位安装包大小为470 Mb, 是miniconda的扩展, 包含了数据科学和机器学习要用到的很多软件


\begin{frame}[standout]{安装和配置conda环境}
    下载: tuna → conda → 清华大学 tuna 镜像站 → 搜索 conda → 选择 anaconda → 选择 miniconda → 选择``latest''版本!
    \begin{myoutline}
        \1 Windows
        \1 MacOS
        \1 Linux
        \1 用法
            \2 conda init 初始化 Shell 环境
            \2 创建一个指定 Python 版本的Conda环境
            \2 切换不同 Python 版本的Conda环境
            \2 对不同Conda环境安装不同版本的 Python 包
            \2 对一个Conda环境安装其他软件
    \end{myoutline}
    \footnotenoindex{https://conda.io/projects/conda/en/latest/user-guide/install/index.html}
    \footnotenoindex{https://mirrors.tuna.tsinghua.edu.cn/anaconda/archive/}
    \footnotenoindex{https://mirrors.tuna.tsinghua.edu.cn/anaconda/miniconda/}
\end{frame}

\begin{frame}[standout]安装和配置conda环境--Windows\end{frame}
\begin{frame}[standout]安装和配置conda环境--MacOS\end{frame}
\begin{frame}[standout]安装和配置conda环境--Linux\end{frame}

% Windows
% 选为所有人安装
% 加入环境变量!!!一定要选
% 注册 anaconda 为系统 Python!!
% install, next, finish
% myPC 属性 about 高级系统设置 环境变量



\begin{frame}[fragile]{Conda源--以MacOS 为例}
    \begin{columns}
        \column{0.82\textwidth}
        \begin{lstlisting}
# 清除之前残留的conda channels
rm -rf ~/.condarc
# 按顺序依次添加channel, 尽可能使用官方源
# 其他源常常在同步库的时候发生
# md5 值校验错误装不上包的问题!!!!
conda config --add channels defaults
conda config --add channels conda-forge
conda config --add channels bioconda
conda config --add channels r
# 备用, 清华源
# https://mirrors.tuna.tsinghua.edu.cn/anaconda/cloud/
# https://mirrors.tuna.tsinghua.edu.cn/anaconda/pkgs/
conda config --add channels https://mirrors.tuna.tsinghua.edu.cn/anaconda/cloud/msys2/
conda config --add channels https://mirrors.tuna.tsinghua.edu.cn/anaconda/cloud/conda-forge
conda config --add channels https://mirrors.tuna.tsinghua.edu.cn/anaconda/pkgs/free/
conda config --add channels https://mirrors.tuna.tsinghua.edu.cn/anaconda/pkgs/main/
#设置搜索是显示通道地址
conda config --set show_channel_urls yes
#查看当前conda配置
conda config --show channels
conda config --get channels
        \end{lstlisting}
        \column{0.12\textwidth}
        \begin{lstlisting}
cat ~/.condarc
# return
channels:
    - r
    - bioconda
    - conda-forge
    - microsoft
    - defaults
        \end{lstlisting}
    \end{columns}
\end{frame}


\begin{frame}[fragile]{pip源}
    \begin{lstlisting}
# PyPI 镜像在每次同步成功后间隔 5 分钟同步一次。
# 临时使用, 注意,simple 不能少, 是 https 而不是 http
pip install -i https://pypi.tuna.tsinghua.edu.cn/simple some-package

# 设为默认
# 升级 pip 到最新的版本 (>=10.0.0) 后进行配置:
python -m pip install --upgrade pip
pip config set global.index-url https://pypi.tuna.tsinghua.edu.cn/simple
# 如果您到 pip 默认源的网络连接较差,临时使用本镜像站来升级 pip:
python -m pip install -i https://pypi.tuna.tsinghua.edu.cn/simple --upgrade pip
    \end{lstlisting}
    \footnotenoindex{https://mirrors.tuna.tsinghua.edu.cn/anaconda/}
\end{frame}


\begin{frame}[fragile]{Conda, Pip的扩展用例}
    \begin{lstlisting}
# install mamba
conda install -c conda-forge mamba
mamba update -n base -c defaults conda
mamba install pandas
# conda 导出环境/导入环境
## 导出当前环境:
conda env export > requirements_conda.yml
## 导入环境:
conda env create -f requirements_conda.yml
## 备注:一些找不到的小组件直接删掉后再安装
# 滚回某个环境
conda list --revisions
conda install --revision=n
# 删除某环境
conda env remove -n learn

# pip 导出环境/导入环境
## 导出当前环境:
pip freeze > requirements.txt
## 导入环境:
pip install -r requirements.txt
    \end{lstlisting}
\end{frame}